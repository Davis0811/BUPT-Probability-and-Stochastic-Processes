% BUPT-Probability-and-Stochastic-Processes (c) by Zhenglong WU(itdevwu)

% BUPT-Probability-and-Stochastic-Processes has a Chinese name "概率论与随机过程习题解答".

% BUPT-Probability-and-Stochastic-Processes is licensed under a
% Creative Commons Attribution-ShareAlike 4.0 International License.

% You should have received a copy of the license along with this
% work. If not, see <http://creativecommons.org/licenses/by-sa/4.0/>.

\chapter{Events and Their Probabilities}
\newpage % 由于此前设置 chapter 标题后不换页,此处换页

\section{Exercises Solution}

\subsection{Experiment, Sample Space and Random Event}

% 1.1
\paragraph{1.1}
We consider the following random experiment: a fair die is rolled; if (and only if) a 6 is obtained, the die is rolled a second time. How many elementary outcomes are there in the sample space $\Omega$?
 
\solution
If the first roll obtained $k\in \{1,2,3,4,5\}$, \par
than the result set would be $\mathbf{A} = \{1,2,3,4,5\}$,\par
else, there would be a second roll that obviously have 6 results.\par
Thus, $card(\Omega) = card(\mathbf{A}) + 6 = 11$.

% 1.2
\paragraph{1.2}
An academic department has just completed voting by secret ballot for a department head. The ballot box contains four slips with votes for candidate A and three slips with votes for candidate B. Suppose that these slips are removed from the box one by one.
\begin{itemize}
    \item[(a)] List all possible outcomes.
    \item[(b)] Suppose that a running tally is kept as slips are removed. For what outcomes does A remain ahead of B throughout the tally?
\end{itemize}

\solution
(a) Consider combinations and permutations:

When slips for B are continuous: $C_{5}^{1}$.

When 2 of slips for B are continuous and the other is not: $A_{5}^{2}$.

When none of slips for B are not continuous: $C_{5}^{3}$.

Thus there should be $C_{5}^{1}+A_{5}^{2}+C_{5}^{3} = 5+20+10 = 35$ outcomes:
\begin{equation*}
    \begin{split}
      \{& AAAABBB, AAABABB, AAABBAB, AAABBBA, AABAABB,\\
        & AABABAB, AABABBA, AABBAAB, AABBABA, AABBBAA,\\
        & ABAAABB, ABAABAB, ABAABBA, ABABAAB, ABABABA,\\
        & ABABBAA, ABBAAAB, ABBAABA, ABBABAA, ABBBAAA,\\
        & BAAAABB, BAAABAB, BAAABBA, BAABAAB, BAABABA,\\
        & BAABBAA, BABAAAB, BABAABA, BABABAA, BABBAAA,\\
        & BBAAAAB, BBAAABA, BBAABAA, BBABAAA, BBBAAAA\}
    \end{split}
\end{equation*}

(b)
$\{AAAABBB, AAABBAB, AABAABB, AABABAB, AABABBA\}$

% 1.3
\paragraph{1.3}
From 10 married couples, we want to select a group of people that is not allowed to contain a married couple.
\begin{itemize}
    \item[(a)] How many choices are there?
    \item[(b)] How many choices are there if the group must also consist of 3 men and 3 women?
\end{itemize}

\solution
(a)

Select 6 couples out of 10, then select 1 person from each couple.

Thus we have $C_{10}^6 {(C_{2}^1)}^6 = 210 \times 64 = 13440$.

(b)

Firstly we select 6 couples out of 10.

For "lady first", we select 3 couples out of previous 6, then select 1 female from each.

Other couples will be selected to choose male.

Therefore there will be nothing to choose from them.

Thus the answer is $C_{10}^6 C_{6}^3 = 210 \times 20 = 4200$.

% 1.4
\paragraph{1.4}
Consider n-digit numbers where each digit is one of the 10 integers $0, 1, \cdots ,9$. How many such numbers are there for which
\begin{itemize}
    \item[(a)] no two consecutive digits are equal?
    \item[(b)] 0 appears as a digit a total of $i$ times $i = 0,\cdots ,n?$
\end{itemize}

\solution
(a)

First digit must not be 0 and any digit cannot be the same as the previous one.

Thus the answer is ${(C_9^1)}^{10} = 9^{10} = 3486784401$.

(b)

Consider 0 appears $i$ times:

We select $i$ positions out of 9 for 0, the rest positions can be any digit except 0.

Thus we get $C_9^1 C_9^i (C_{9}^{1})^{9-i} = 9 C_9^i \cdot 9^{9-i} = C_9^i \cdot 9^{10-i}$ as answer.

(Calculation)

We can use Python or other numeric computing methods to calculate such formula.

\textit{\textbf{Note}: There is a possibility that the answer exceeds INT\_MAX during calculation.}

% 1.4的答案,允许在此处或底部浮动
\begin{table}[hb]
    \centering
    \resizebox{.95\textwidth}{!}{
        \begin{tabular}{|c|c|c|c|c|c|c|c|c|c|c|}
            \hline
            \textit{\textbf{i}}     & 0          & 1          & 2          & 3         & 4        & 5       & 6      & 7     & 8   & 9 \\ \hline
            \textit{\textbf{count}} & 3486784401 & 3486784401 & 1549681956 & 401769396 & 66961566 & 7440174 & 551124 & 26244 & 729 & 9 \\ \hline
        \end{tabular}
    }
\end{table}

\paragraph{1.5}
Let E,F,G be three events. Find expressions for the events that of E, G, F
\begin{itemize}
    \item[(a)] only F occurs,
    \item[(b)] both E and F but not G occur,
    \item[(c)] at least one event occurs,
    \item[(d)] at least two events occur,
    \item[(e)] all three events occur,
    \item[(f)] none occurs,
    \item[(g)] at most one occurs,
    \item[(h)] at most two occur.
\end{itemize}

\noindent \textit{\textbf{Notation}: Given a set $U$, $B$ is complement of $A$ in U, written $B = A \setminus U = A^C = U - A$.}

\solution

For two events $A, B$, there are:

\begin{center}
    $A\cup B$ is an event in which A OR B (or both) occur,\\
    $A\cap B = AB$ is an event in which A AND B occur,\\
    $A^C$ is an event in which A does not occur.
\end{center}

\begin{itemize}
    \item[(a)] $F[(E\cup G)^C]$
    \item[(b)] $(E\cup F)(G^C)$
    \item[(c)] $E\cup F\cup G$
    \item[(d)] $(EF)\cup (EG)\cup (FG)$
    \item[(e)] $EFG$
    \item[(f)] $(E\cup F\cup G)^C$
    \item[(g)] $[(EF)\cup (EG)\cup (FG)]^C$
    \item[(h)] $(EFG)^C$
\end{itemize}

\textit{\textbf{Note}: There are relationships between problem (c, d, e) \& problem (f, g, h).}
