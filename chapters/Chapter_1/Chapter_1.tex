% BUPT-Probability-and-Stochastic-Processes (c) by Zhenglong WU(itdevwu)

% BUPT-Probability-and-Stochastic-Processes has a Chinese name "概率论与随机过程习题解答".

% BUPT-Probability-and-Stochastic-Processes is licensed under a
% Creative Commons Attribution-ShareAlike 4.0 International License.

% You should have received a copy of the license along with this
% work. If not, see <http://creativecommons.org/licenses/by-sa/4.0/>.

\chapter{Events and Their Probabilities}
\newpage % 由于此前设置 chapter 标题后不换页,此处换页

\section{Exercises Solution}

\subsection{Experiment, Sample Space and Random Event}

% 1.1
\paragraph{1.1}
We consider the following random experiment: a fair die is rolled; if (and only if) a 6 is obtained, the die is rolled a second time. How many elementary outcomes are there in the sample space $\Omega$?
 
\begin{solution}

If the first roll obtained $k\in \{1,2,3,4,5\}$, \par
than the result set would be $\mathbf{A} = \{1,2,3,4,5\}$,\par
else, there would be a second roll that obviously have 6 results.\par
Thus, $card(\Omega) = card(\mathbf{A}) + 6 = 11$.

\end{solution}

% 1.2
\paragraph{1.2}
An academic department has just completed voting by secret ballot for a department head. The ballot box contains four slips with votes for candidate A and three slips with votes for candidate B. Suppose that these slips are removed from the box one by one.
\begin{itemize}
    \item[(a)] List all possible outcomes.
    \item[(b)] Suppose that a running tally is kept as slips are removed. For what outcomes does A remain ahead of B throughout the tally?
\end{itemize}

\begin{solution}

(a) When slips for B are continuous: $C_{5}^{1}$.

When 2 of slips for B are continuous and the other is not: $A_{5}^{2}$.

When none of slips for B are not continuous: $C_{5}^{3}$.

Thus there should be $C_{5}^{1}+A_{5}^{2}+C_{5}^{3} = 5+20+10 = 35$ outcomes:
\begin{equation*}
    \begin{split}
      \{& AAAABBB, AAABABB, AAABBAB, AAABBBA, AABAABB,\\
        & AABABAB, AABABBA, AABBAAB, AABBABA, AABBBAA,\\
        & ABAAABB, ABAABAB, ABAABBA, ABABAAB, ABABABA,\\
        & ABABBAA, ABBAAAB, ABBAABA, ABBABAA, ABBBAAA,\\
        & BAAAABB, BAAABAB, BAAABBA, BAABAAB, BAABABA,\\
        & BAABBAA, BABAAAB, BABAABA, BABABAA, BABBAAA,\\
        & BBAAAAB, BBAAABA, BBAABAA, BBABAAA, BBBAAAA\}
    \end{split}
\end{equation*}

(b)
$\{AAAABBB, AAABBAB, AABAABB, AABABAB, AABABBA\}$

\end{solution}

% 1.3
\paragraph{1.3}
From 10 married couples, we want to select a group of people that is not allowed to contain a married couple.
\begin{itemize}
    \item[(a)] How many choices are there?
    \item[(b)] How many choices are there if the group must also consist of 3 men and 3 women?
\end{itemize}

\begin{solution}

(a)

Select 6 couples out of 10, then select 1 person from each couple.

Thus we have $C_{10}^6 {(C_{2}^1)}^6 = 210 \times 64 = 13440$.

(b)

Firstly we select 6 couples out of 10.

For "lady first", we select 3 couples out of previous 6, then select 1 female from each.

Other couples will be selected to choose male.

Therefore there will be nothing to choose from them.

Thus the answer is $C_{10}^6 C_{6}^3 = 210 \times 20 = 4200$.

\end{solution}

% 1.4
\paragraph{1.4}
Consider n-digit numbers where each digit is one of the 10 integers $0, 1, \cdots ,9$. How many such numbers are there for which
\begin{itemize}
    \item[(a)] no two consecutive digits are equal?
    \item[(b)] 0 appears as a digit a total of $i$ times $i = 0,\cdots ,n?$
\end{itemize}

\begin{solution}

(a) First digit must not be 0 and any digit cannot be the same as the previous one.

Thus the answer is ${(C_9^1)}^{10} = 9^{10} = 3486784401$.

(b) Consider 0 appears $i$ times:

We select $i$ positions out of 9 for 0, the rest positions can be any digit except 0.

Thus we get $C_9^1 C_9^i (C_{9}^{1})^{9-i} = 9 C_9^i \cdot 9^{9-i} = C_9^i \cdot 9^{10-i}$ as answer.

(Calculation)

We can use Python or other numeric computing methods to calculate such formula.

\textit{\textbf{Note}: There is a possibility that the answer exceeds INT\_MAX during calculation.}

\end{solution}

% 1.4的答案,允许在此处或底部浮动
\begin{table}[hb]
    \centering
    \resizebox{\textwidth}{!}{
        \begin{tabular}{|c|c|c|c|c|c|c|c|c|c|c|}
            \hline
            \textit{\textbf{i}}     & 0          & 1          & 2          & 3         & 4        & 5       & 6      & 7     & 8   & 9 \\ \hline
            \textit{\textbf{count}} & 3486784401 & 3486784401 & 1549681956 & 401769396 & 66961566 & 7440174 & 551124 & 26244 & 729 & 9 \\ \hline
        \end{tabular}
    }
\end{table}

% 1.5
\paragraph{1.5}
Let E,F,G be three events. Find expressions for the events that of E, G, F
\begin{itemize}
    \item[(a)] only F occurs,
    \item[(b)] both E and F but not G occur,
    \item[(c)] at least one event occurs,
    \item[(d)] at least two events occur,
    \item[(e)] all three events occur,
    \item[(f)] none occurs,
    \item[(g)] at most one occurs,
    \item[(h)] at most two occur.
\end{itemize}

\noindent \textit{\textbf{Notation}: Given a set $U$, $B$ is complement of $A$ in U, written $B = A \setminus U = A^C = U - A$.}

\begin{solution}

For two events $A, B$, there are:

\begin{center}
    $A\cup B$ is an event in which A OR B (or both) occur,\\
    $A\cap B = AB$ is an event in which A AND B occur,\\
    $A^C$ is an event in which A does not occur.
\end{center}

\begin{itemize}
    \item[(a)] $F[(E\cup G)^C]$
    \item[(b)] $(E\cup F)(G^C)$
    \item[(c)] $E\cup F\cup G$
    \item[(d)] $(EF)\cup (EG)\cup (FG)$
    \item[(e)] $EFG$
    \item[(f)] $(E\cup F\cup G)^C$
    \item[(g)] $[(EF)\cup (EG)\cup (FG)]^C$
    \item[(h)] $(EFG)^C$
\end{itemize}

\textit{\textbf{Note}: There are relationships between problem (c, d, e) \& problem (f, g, h).}

\end{solution}

% 1.6
\paragraph{1.6}
Express the following events in the simplest form.
\begin{itemize}
    \item[(a)] $(A\cup B)\cap (B\cup C)$,
    \item[(b)] $(A\cup B)\cap (A\cup \overline{B})$,
    \item[(c)] $(A\cup B)\cap (A\cup \overline{B})\cap (\overline{A}\cup B)$.
\end{itemize}

\begin{solution}

\textit{\textbf{Note}: Some discrete mathematics knowledge will be helpful.\cite{ISBN9781259676512}}

(a)

\begin{equation*}
    (A\cup B)\cap (B\cup C)\iff (A\cap C)\cup B\iff (AC)\cup B\\
\end{equation*}

(b)

\begin{align*}
    & (A\cup B)\cap (A\cup \overline{B})\\
    \iff & A\cup (B\cap \overline{B})\\
    \iff & A\cup \varnothing\\
    \iff & A\\
\end{align*}

(c) From (b) we have:

\begin{align*}
    & (A\cup B)\cap (A\cup \overline{B})\cap (\overline{A}\cup B)\\
    \iff & A\cap (\overline{A}\cup B)\\
    \iff & A\cap B\iff AB\\
\end{align*}

\end{solution}

% 1.7
\paragraph{1.7}
Suppose that a coin is tossed ten times. Let $A$ obtain the event that a head is obtained on the first toss, and let $B$ denote the event that a head is obtained on the sixth toss. Are $A$ and $B$ disjoint?

\begin{solution}

No, they are not disjoint for they can probably occur at the same time.

\end{solution}

\subsection{Probabilities Defined On Events}

% 1.8
\paragraph{1.8}
A box contains three marbles: one red, one green and one blue. Consider an experiment that consists of taking one marble from the box then replacing it in the box and drawing a second marble from the box. What is the sample space? If, at all times, each marble in the box is equally likely to be selected, what is the probability of each point in the sample space?

\begin{solution}

All steps are independent.

Sample space $\Omega = \{(r,r), (r,g), (r,b), (g,r), (g,g), (g,b), (b,r), (b,g), (b,b)\}$.

For each step $P(r) = P(g) = P(b) = \frac{1}{3}$.

Thus, $\forall(x, y)\in \{(x,y):x, y\in\{r, g, b\}\}$, there is:

\begin{equation*}
    P[(x, y)] = P(xy) = P(x)P(y) = \frac{1}{9}
\end{equation*}

\end{solution}

% 1.9
\paragraph{1.9}
You roll two dice. What are the probability of the events:

\begin{itemize}
    \item[(a)] They show the same?
    \item[(b)] Their sum is seven or eleven?
    \item[(c)] They have no common factor greater than unity? (\textit{unity} stands for number \textit{one})
    \item[(d)] The sum of the numbers is 2, 3 or 12?
    \item[(e)] The sum is odd?
    \item[(f)] The difference is odd?
    \item[(g)] The product is odd?
    \item[(h)] One number divides the other?
    \item[(i)] The first die shows a smaller number than the second?
    \item[(j)] Different numbers are shown and the smaller of the two number is $r$, $1\leq r\leq 6$.
\end{itemize}

\begin{solution}
(a) $P=\frac{6}{36}=\frac{1}{6}$.

(b)

For 11, the only possible results are $\{5, 6\}, \{6, 5\}$, so the answer is $P=\frac{2}{36}=\frac{1}{18}$.

For 7, we have $C_{7-1}^{1}=6$, so the answer is $P=\frac{6}{36}=\frac{1}{6}$.

(c)

The two results are co-prime if and only if the circumstance is one of: $\{2, 4\}, \{4, 2\},$ $\{2, 6\}, \{6, 2\}, \{3, 6\}, \{6, 3\}, \{4, 6\}, \{6, 4\}$. Hence the answer is $P=\frac{8}{36}=\frac{2}{9}$.

(d)

The required results are: $\{1, 1\}, \{1, 2\}, \{2, 1\}, \{6, 6\}$. Hence the answer is $P=\frac{4}{36}=\frac{1}{9}$.

(e)

The sum is odd if and only if one result is odd while the other is even. Therefore e have $P=\frac{C_{3}^{1}C_{3}^{1}+C_{3}^{1}C_{3}^{1}}{36}=\frac{1}{2}$.

(f)

It is obvious that the two results can be represented as $x$ and $x+\delta$. Since $2x$ is even and $\delta$ is odd, their sum $2x+\delta$ is odd. Similarly, if the difference is even, their sum is even. Thus, (f) $\iff$ (e) and the answer is the same $P=\frac{1}{2}$.

(g)

The product is odd if and only if all results are odd. Hence we have $P=\frac{C_{3}^{1}C_{3}^{1}}{36}=\frac{1}{4}$.

(h)

One number divides the other if and only if the circumstance is one of: $\{2, 4\}, \{4, 2\},$ $\{2, 6\}, \{6, 2\}, \{3, 6\}, \{6, 3\}$. Hence the answer is $P=\frac{6}{36}=\frac{1}{6}$.

(i)$P=\frac{1}{36}\Sigma_{i=1}^{6}i-1=\frac{5}{12}$.

(j)

$P=\frac{2(6-r)C_{6}^{1}}{36}=\frac{6-r}{3}$.
\end{solution}