% BUPT-Probability-and-Stochastic-Processes (c) by Zhenglong WU(itdevwu)

% BUPT-Probability-and-Stochastic-Processes has a Chinese name "概率论与随机过程习题解答".

% BUPT-Probability-and-Stochastic-Processes is licensed under a
% Creative Commons Attribution-ShareAlike 4.0 International License.

% You should have received a copy of the license along with this
% work. If not, see <http://creativecommons.org/licenses/by-sa/4.0/>.

\chapter{Events and Their Probabilities}
\newpage % 由于此前设置 chapter 标题后不换页,此处换页

\section{Exercises Solution}

\subsection{Experiment, Sample Space and Random Event}

\paragraph{1.1}
We consider the following random experiment: a fair die is rolled; if (and only if) a 6 is obtained, the die is rolled a second time. How many elementary outcomes are there in the sample space $\Omega$?
 
\solution
If the first roll obtained $k\in \{1,2,3,4,5\}$, \par
than the result set would be $\mathbf{A} = \{1,2,3,4,5\}$,\par
else, there would be a second roll that obviously have 6 results.\par
Thus, $card(\Omega) = card(\mathbf{A}) + 6 = 11$.

\paragraph{1.2}
An academic department has just completed voting by secret ballot for a department head. The ballot box contains four slips with votes for candidate A and three slips with votes for candidate B. Suppose that these slips are removed from the box one by one.
\begin{itemize}
    \item[(a)] List all possible outcomes.
    \item[(b)] Suppose that a running tally is kept as slips are removed. For what outcomes does A remain ahead of B throughout the tally?
\end{itemize}

\solution
(a) Consider combinations and permutations:\par
When slips for B are continuous: $C_{5}^{1}$\par
When 2 of slips for B are continuous and the other is not: $A_{5}^{2}$\par
When none of slips for B are not continuous: $C_{5}^{3}$\par
Thus there should be $C_{5}^{1}+A_{5}^{2}+C_{5}^{3} = 5+20+10 = 35$ outcomes:
\begin{equation*}
    \begin{split}
      \{& AAAABBB, AAABABB, AAABBAB, AAABBBA, AABAABB,\\
        & AABABAB, AABABBA, AABBAAB, AABBABA, AABBBAA,\\
        & ABAAABB, ABAABAB, ABAABBA, ABABAAB, ABABABA,\\
        & ABABBAA, ABBAAAB, ABBAABA, ABBABAA, ABBBAAA,\\
        & BAAAABB, BAAABAB, BAAABBA, BAABAAB, BAABABA,\\
        & BAABBAA, BABAAAB, BABAABA, BABABAA, BABBAAA,\\
        & BBAAAAB, BBAAABA, BBAABAA, BBABAAA, BBBAAAA\}
    \end{split}
\end{equation*}\par

(b)
$\{AAAABBB, AAABBAB, AABAABB, AABABAB, AABABBA\}$